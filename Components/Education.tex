\section*{Education}
\begin{longtable}{L!{\VRule}R}
2010--2016&\textbf{PhD} in Electronic and Electrical Engineering\\
& \textit{New Algorithms for Ultrasonic Non-Destructive Testing} \\[5pt]
& This research was inspired by a growing necessity to verify the structural integrity of industrial components. Throughout this study, I worked closely with large multinational companies including Amec Foster Wheeler, Royal Dutch Shell and Rolls Royce.\\[5pt]
& New signal processing algorithms were developed that operate on both time-series and frequency-domain signals and improve signal-to-noise ratio of ultrasonic imaging by over 2.5x compared to the current state-of-the-art.\\[5pt]
& A new efficient imaging algorithm was also developed that allowed the rapid processing of ultrasonic data using CUDA C++ for parallel processing. The resulting algorithm was shown to outperform the next closest competitor by 50x at the time of publication.\\[5pt]
& I also gained experience into distributed computing and have worked with Oracle Grid Engine. \\ [5pt]
& Multiple publications have arisen from this research and the work from this study has been presented at two international conferences as well as a number of national engagements.\\[5pt]
2005--2010&\textbf{MEng} in Electronic and Electrical Engineering\\[5pt]
&I studied a number of communications, microcontrollers and digital signal processing modules towards culmination of my undergraduate degree. I gained experience in digital communications networks, both wired and wireless. I using C, C++, Assembly and MATLAB on a combination of OS X, Windows, Linux and embedded devices.\\[5pt]
&\textit{Projects}\\[5pt]
&An investigation was made into the operation of WiMAX. Throughpu t of a network was tested using different modulation schemes such as BPSK and QAM. Wireless data was recorded using an Agilent VXI mainframe configured with signal-capture hardware. Vector signal analysis software was used to decode and interpret recorded WiMAX data. A vector signal generator was used to inject noise into the system to test the resilience of the network. \\[15pt]
&A home automation system was developed using Texas Instruments MSP430 microcontroller-based keychains to track users' movements throughout an area via the RSSI protocol over Zigbee. Location information was fed to a C++ application running on a Linux server which relayed instructions to a series of Microchip PIC microcontrollers via a Lantronix XPort ethernet to serial adapter. The digital IO of the PIC microcontrollers were used to control relays connected to light switches and sockets to allow appliances to react to the movement of a user throughout the home.\\

\end{longtable}
