\section*{Teaching Experience}
\begin{longtable}{L!{\VRule}R}
2011--Present&\textbf{Engineering Design for Software Development}\\[5pt]
&I teach Python to freshman Electronic and Electrical Engineering students. This is often the first time these students are exposed to programming. The foundations of computer programming are taught in the first semester and practical exercises are set in labs. The second semester sees groups of students attempt a challenging programming project where the students are encouraged to research and develop their own solutions to the problems they encounter.\\[5pt]
2015--Present&\textbf{Individual Project}\\[5pt]
&I supervise students as they undertake a their final-year project to complete their honours degree. This involves meeting regularly with the student, recommending literature and working with them to set achievable goals for their project.\\[5pt]
2012--2014&\textbf{Electronic and Electrical Techniques and Design}\\[5pt]
&I take part in a laboratory module for this class that requires the students to design and build an ultrasonic depth measurement device. This involves educating the students on the three key components of this device: the transistor amplifier, the receiving operational amplifier and signal rectification circuit, and the integrated circuit chips for the timer and seven-segment display controller used to show the measured distance.\\[5pt]
2011--2015&\textbf{Instrumentation and Microcontrollers}\\[5pt]
&I taught the laboratory module for this class which involves the students applying their knowledge of C to embedded devices, specifically a Renesas M16C microcontroller. While students have a knowledge of programming at this stage in their studies, it is their first experience with embedded devices and low-level programming. Students are encouraged to read the datasheets for their microcontrollers and exploit the low-level functions such as interrupts, timers and ADCs.\\[5pt]
\end{longtable}