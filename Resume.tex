\documentclass[12pt]{article}
\usepackage[margin=1.5cm]{geometry}
\usepackage{array, xcolor}
\usepackage{longtable}
\usepackage{hyperref}
\definecolor{lightgray}{gray}{0.8}
\newcolumntype{L}{>{\raggedleft}p{0.2\textwidth}}
\newcolumntype{R}{p{0.74\textwidth}}
\newcommand\VRule{\color{lightgray}\vrule width 0.5pt}
\newcommand{\tabitem}{~~\llap{\textbullet}~~}
\let\stdsection\section
\title{\bfseries\Huge Timothy Lardner, PhD}
\author{tim@lardner.io}
\date{}
\begin{document}
\maketitleı
\section*{Overview}
I am a highly-motivated research scientist and engineer from the United Kingdom. My research focus is currently ultrasonic signal processing and automation for industrial non-destructive testing.\\[5pt]

I'm looking for a new challenging position that will allow me to explore novel solutions to engineering problems, both hardware and software.\\[5pt]

In my current position, I have improved both the accuracy and robustness of an engineering process while reducing the processing time by 98\%. I work within the the University of Strathclyde's Advanced Nuclear Research Centre where my work has been presented to industry leaders.\\[5pt]

I am proficient in MATLAB, LabVIEW, Python, C, C++, and CUDA with experience in Java, C\# and other languages. I use OS X, Windows and Linux systems on a daily basis and am comfortable working with these.\\[5pt]

I enjoy programming and solving problems, both in the office and in my leisure time. Some of my projects are available at \url{https://github.com/timlardner}.


\renewcommand\section{\newpage\stdsection}
\section*{Professional Experience}
\begin{longtable}{L!{\VRule}R}
2017--Present&JP Morgan Chase \& Co: \textbf{Software Engineer}\\[5pt]
&Python, Django, Javascript, AngularJS, REST, Agile, Scrum, Application Security, Atlassian workflow tools, Git\\[5pt]

2014--2017&University of Strathclyde: \textbf{Research Associate}\\[5pt]
& I within the Centre for Ultrasonic Engineering (CUE) which is a multidisciplinary research environment. I work on a daily basis with electrical engineers, mechanical engineers, mathematicians, physicists, chemists and biologists in order to solve engineering challenges and problems. Part of my duties involve introducing industrial and academic visitors and sponsors to the group and providing an overview of both my and the group's research.\\[5pt]
& \textit{Projects}\\[5pt]
&I managed the development of a software package called `cueART' which is a LabVIEW based frontend for the acquisition and processing of ultrasonic data. This software interacts with ultrasonic phased-array controllers using both LabVIEW and C\# plugins that were also developed within CUE.\\[5pt]

&The data was processed using CUDA C++ based algorithms developed both by myself and other researchers within CUE as part of their standard duties.  The motivation for this work was to make these algorithms available for everyone in the group to use.\\[5pt]

&This software was then expanded to integrate Kuka KR5 and KR90 robotics systems via a TCP/IP socket for data acquisition and metrology metadata.\\[5pt]

&Challenges of this project included ensuring metrology accuracy when fusing the robotic positioning data with the data recorded from the ultrasonic probes.\\[15pt]

&I am working to develop a system for the automation of processing of ultrasonic data from pressure tubes within CANDU nuclear reactors. This involves working closely with analysts for knowledge solicitation. The analysts' procedures are recorded and deviations from the acknowledged specification are identified. These procedures are codified as an algorithm and applied to ultrasonic datasets. The algorithms employ both signal and image processing techniques, as well as incorporating machine learning methodologies to classify any defects found within the pressure tubes. \\[5pt]

&As it currently stands, the new algorithm allows for a 98\% reduction in processing time compared to the time taken for analysts to process the data manually and efforts are being made to incorporate the developed algorithms into the analysts' workflow for future inspections.\\[5pt]

2010&Cable \& Wireless: \textbf{Corporate Faults Adviser}\\[5pt]
&I worked for Cable \& Wireless within the corporate faults department where I was the first contact for businesses with connectivity problems. Within this post, I advanced to the Global Infrastructure department which involved working with Network Operations Centres to locate and diagnose faults on major circuits throughout the world. This involved a large amount of organisation as well as being able to communicate effectively with customers and team members worldwide.
\end{longtable}

\section*{Education}
\begin{longtable}{L!{\VRule}R}
2010--2016&\textbf{PhD} in Electronic and Electrical Engineering\\
&\textit{New Algorithms for Ultrasonic Non-Destructive Testing} \\[5pt]
&This research was inspired by a growing necessity to verify the structural integrity of industrial components. Throughout this study, I worked on projects funded by large multinational companies including Amec Foster Wheeler, Royal Dutch Shell and Rolls Royce.\\[5pt]
& New signal processing algorithms were developed that operate on both time and frequency domain signals and improve signal-to-noise ratio of ultrasonic imaging by over 2.5x compared to the current state-of-the-art.\\[5pt]
&A new efficient imaging algorithm was also developed that allowed the rapid processing of ultrasonic data using CUDA C++ for parallel processing. The resulting algorithm was shown to outperform the next closest competitor by 50x at the time of publication.\\[5pt]
&Multiple publications have arisen from this research and the work from this study has been presented at two international conferences as well as a number of national engagements.\\[5pt]
2005--2010&\textbf{MEng} in Electronic and Electrical Engineering\\[5pt]
&I studied a number of communications, microcontrollers and digital signal processing modules towards culmination of my undergraduate degree. I gained experience in digital communications networks, both wired and wireless. I programmed using C, C++, Assembly and MATLAB on a combination of OS X, Windows, Linux and embedded devices.\\[5pt]
&\textit{Projects}\\[5pt]
&An investigation was made into the operation of WiMAX. Throughput of a network was tested using different modulation schemes such as BPSK and QAM. Wireless data was recorded using an Agilent VXI mainframe configured with signal-capture hardware. Vector signal analysis software was used to decode and interpret recorded WiMAX data. A vector signal generator was used to inject noise into the system to test the resilience of the network. \\[15pt]
&A home automation system was developed using Texas Instruments MSP430 microcontroller-based keychains to track users' movements throughout an area via the RSSI protocol over Zigbee. Location information was fed to a C++ application running on a Linux server which relayed instructions to a series of Microchip PIC microcontrollers via a Lantronix XPort ethernet to serial adapter. The digital IO of the PIC microcontrollers were used to control relays connected to light switches and sockets to allow appliances to react to the movement of a user throughout the home.\\

\end{longtable}

\section*{Teaching Experience}
\begin{longtable}{L!{\VRule}R}
2011--Present&\textbf{Engineering Design for Software Development}\\[5pt]
&I teach Python to freshman Electronic and Electrical Engineering students. This is often the first time these students are exposed to programming. The foundations of computer programming are taught in the first semester and practical exercises are set in labs. The second semester sees groups of students attempt a challenging programming project where the students are encouraged to research and develop their own solutions to the problems they encounter.\\[5pt]
2015--Present&\textbf{Individual Project}\\[5pt]
&I supervise students as they undertake a their final-year project to complete their honours degree. This involves meeting regularly with the student, recommending literature and working with them to set achievable goals for their project.\\[5pt]
2012--2014&\textbf{Electronic and Electrical Techniques and Design}\\[5pt]
&I take part in a laboratory module for this class that requires the students to design and build an ultrasonic depth measurement device. This involves educating the students on the three key components of this device: the transistor amplifier, the receiving operational amplifier and signal rectification circuit, and the integrated circuit chips for the timer and seven-segment display controller used to show the measured distance.\\[5pt]
2011--2015&\textbf{Instrumentation and Microcontrollers}\\[5pt]
&I taught the laboratory module for this class which involves the students applying their knowledge of C to embedded devices, specifically a Renesas M16C microcontroller. While students have a knowledge of programming at this stage in their studies, it is their first experience with embedded devices and low-level programming. Students are encouraged to read the datasheets for their microcontrollers and exploit the low-level functions such as interrupts, timers and ADCs.\\[5pt]
\end{longtable}

\bibliographystyle{unsrt}
\renewcommand{\refname}{List of Publications}
\nocite{*}
\bibliography{Me}


\end{document}
